\documentclass[11pt]{article}

% Change "review" to "final" to generate the final (sometimes called camera-ready) version.
% Change to "preprint" to generate a non-anonymous version with page numbers.
\usepackage[]{acl}
\usepackage{times}
\usepackage{latexsym}

% For proper rendering and hyphenation of words containing Latin characters (including in bib files)
\usepackage[T1]{fontenc}
% For Vietnamese characters
% \usepackage[T5]{fontenc}
% See https://www.latex-project.org/help/documentation/encguide.pdf for other character sets

% This assumes your files are encoded as UTF8
\usepackage[utf8]{inputenc}
\usepackage{microtype}
\usepackage{inconsolata}
\usepackage{graphicx}

% If the title and author information does not fit in the area allocated, uncomment the following
%
%\setlength\titlebox{<dim>}
%
% and set <dim> to something 5cm or larger.

\title{Group X Progress Report:\\My Group's Project Name}


\author{Arjun Karthik, Alina Zeng, Stanley Chen \\
  \texttt{\{macid1,macid2,macid3\}@mcmaster.ca} }

%\author{
%  \textbf{First Author\textsuperscript{1}},
%  \textbf{Second Author\textsuperscript{1,2}},
%  \textbf{Third T. Author\textsuperscript{1}},
%  \textbf{Fourth Author\textsuperscript{1}},
%\\
%  \textbf{Fifth Author\textsuperscript{1,2}},
%  \textbf{Sixth Author\textsuperscript{1}},
%  \textbf{Seventh Author\textsuperscript{1}},
%  \textbf{Eighth Author \textsuperscript{1,2,3,4}},
%\\
%  \textbf{Ninth Author\textsuperscript{1}},
%  \textbf{Tenth Author\textsuperscript{1}},
%  \textbf{Eleventh E. Author\textsuperscript{1,2,3,4,5}},
%  \textbf{Twelfth Author\textsuperscript{1}},
%\\
%  \textbf{Thirteenth Author\textsuperscript{3}},
%  \textbf{Fourteenth F. Author\textsuperscript{2,4}},
%  \textbf{Fifteenth Author\textsuperscript{1}},
%  \textbf{Sixteenth Author\textsuperscript{1}},
%\\
%  \textbf{Seventeenth S. Author\textsuperscript{4,5}},
%  \textbf{Eighteenth Author\textsuperscript{3,4}},
%  \textbf{Nineteenth N. Author\textsuperscript{2,5}},
%  \textbf{Twentieth Author\textsuperscript{1}}
%\\
%\\
%  \textsuperscript{1}Affiliation 1,
%  \textsuperscript{2}Affiliation 2,
%  \textsuperscript{3}Affiliation 3,
%  \textsuperscript{4}Affiliation 4,
%  \textsuperscript{5}Affiliation 5
%\\
%  \small{
%    \textbf{Correspondence:} \href{mailto:email@domain}{email@domain}
%  }
%}

\begin{document}
\maketitle
% \begin{abstract}
% \end{abstract}

\section{Introduction}

This project aims to predict the weekly occurrence of major crime types across Los Angeles using publicly available data from the Los Angeles Police Department. Predicting crime patterns is a valuable tool for law enforcement and city planners, as it supports proactive decision-making, effective resource allocation, and early detection of emerging hot spots. The model developed in this work predicts whether a given crime type will occur in a particular region of the city during a specific week.
Our approach frames the problem as a multi-label binary classification task, where each label represents a different crime type. By incorporating short-term historical data through rolling temporal features, we enable the model to capture recent trends in crime dynamics. The focus is on establishing a clear and explainable baseline model that accurately captures temporal dependencies while serving as a foundation for more advanced methods.

\section{Related Work}

Here, talk about the related work you encountered for your approach. Cite at least 5 references. Refer to item 2. No one has done exactly your task? Write about the most similar thing you can find. This should be around 0.25-0.5 pages.

\section{Dataset}

% You should write about your dataset here, following the guidelines regarding item 1. This section may be 0.5-1 pages. Depending on your specific dataset, you may want to include subsections for the preprocessing, annotation, etc.

The raw dataset consisted of over 1 million individual crime reports, transcribed from the original paper records. Each report included a date, time of occurrence, location, crime code, victim age, and other logistical attributes.

To create a dataset suitable for the intended model, however, several restructuring steps were necessary.

\subsection{Preprocessing}
\begin{enumerate}
    \item Normalizing all column names in the provided \texttt{.csv} file, including aligning capitalization and trimming spaces.
    \item Dropping rows missing a time, date, or location, since these were essential identifiers.
    \item Dropping rows with coordinates outside the latitude and longitude bounds around Los Angeles (LA) (including a buffer margin).
\end{enumerate}

\subsection{Base Transformations}
\begin{enumerate}
    \item Spatial aggregation: The LA area (including a buffer margin) was divided into 2 km$^2$ grid cells, with each cell treated as a single ``spatial unit.''
    \item Temporal aggregation: Crimes were grouped by week number and year. For example, crimes occurring on Tuesday and Wednesday of the 3rd week of a year would belong to the same ``temporal unit.''
    \item Count aggregation: The combined spatial-temporal unit (grid cell $\times$ week) was used to count the number of crimes per type.
\end{enumerate}

\subsection{Add-on Transformations}
\begin{enumerate}
    \item Rolling averages: For each grid cell-week unit, the rolling average of each crime type over the previous two weeks was calculated.
\end{enumerate}

After preprocessing and transformation, the final dataset consisted of grid cell-week units, each with counts for every crime type. Planned extensions include incorporating rolling averages from neighboring grids to capture local spatial trends.

\section{Features}

Describe any features you used for your model, or how your data was input to your model. Are you doing feature engineering or feature selection? Are you learning embeddings? Is it all part of one neural network? Refer to item 2. This may range from 0.25 pages to 0.5 pages.

\section{Implementation}

The model was implemented as a multi-label logistic regression classifier using only NumPy and pandas. Each crime type is modeled as a separate binary logistic regression problem with its own weight vector and bias term. During training, the model learns to minimize the binary cross-entropy loss between the predicted probabilities and the true labels. The optimization is carried out using batch gradient descent with a fixed learning rate and a specified number of iterations.
The logistic regression was trained on rolling temporal features extracted from the training data, and predictions were generated for each week in the testing set. The output probabilities were converted into binary predictions using a 0.5 threshold. Accuracy was computed individually for each crime label by comparing predicted outcomes with actual occurrences. The model’s results and learned parameters were stored for further analysis and visualization.
This implementation offers a balance between how interpretable it is and predictive power. By avoiding reliance on high-level machine learning libraries, it provides clear insight into the mathematical foundations of logistic regression and allows the team to directly control every aspect of the training and evaluation process.

\section{Results and Evaluation}

How are you evaluating your model? What results do you have so far? What are your baselines? Refer to item 5. This may take around 0.5 pages.

\section{Feedback and Plans}

Based on the feedback received, the next stage of this project will focus on expanding the modeling approach and improving both performance and interpretability. The team plans to experiment with different machine learning models, such as Random Forests and neural networks, to compare their predictive capabilities against the current logistic regression baseline. Regularization techniques, including L1 and L2 penalties, will be explored to improve generalization and reduce overfitting. Additionally, the feature preprocessing pipeline will be revisited to test alternative grouping strategies, such as aggregating related crime categories into broader classes to capture higher-level behavioral patterns. The team also aims to develop a spatial visualization component, creating an interactive map to display predicted crime hotspots across Los Angeles. This will improve interpretability and provide an intuitive way to analyze temporal and spatial crime trends. In general, these next steps will strengthen the performance of the model and present the results in a more informative and practical manner.

\section{Template Notes}

You can remove this section or comment it out, as it only contains instructions for how to use this template. You may use subsections in your document as you find appropriate.

\subsection{Tables and figures}

See Table~\ref{citation-guide} for an example of a table and its caption.
See Figure~\ref{fig:experiments} for an example of a figure and its caption.


\begin{figure}[t]
  \includegraphics[width=\columnwidth]{example-image-golden}
  \caption{A figure with a caption that runs for more than one line.
    Example image is usually available through the \texttt{mwe} package
    without even mentioning it in the preamble.}
  \label{fig:experiments}
\end{figure}

\begin{figure*}[t]
  \includegraphics[width=0.48\linewidth]{example-image-a} \hfill
  \includegraphics[width=0.48\linewidth]{example-image-b}
  \caption {A minimal working example to demonstrate how to place
    two images side-by-side.}
\end{figure*}


\subsection{Citations}

\begin{table*}
  \centering
  \begin{tabular}{lll}
    \hline
    \textbf{Output}           & \textbf{natbib command} & \textbf{ACL only command} \\
    \hline
    \citep{Gusfield:97}       & \verb|\citep|           &                           \\
    \citealp{Gusfield:97}     & \verb|\citealp|         &                           \\
    \citet{Gusfield:97}       & \verb|\citet|           &                           \\
    \citeyearpar{Gusfield:97} & \verb|\citeyearpar|     &                           \\
    \citeposs{Gusfield:97}    &                         & \verb|\citeposs|          \\
    \hline
  \end{tabular}
  \caption{\label{citation-guide}
    Citation commands supported by the style file.
  }
\end{table*}

Table~\ref{citation-guide} shows the syntax supported by the style files.
We encourage you to use the natbib styles.
You can use the command \verb|\citet| (cite in text) to get ``author (year)'' citations, like this citation to a paper by \citet{Gusfield:97}.
You can use the command \verb|\citep| (cite in parentheses) to get ``(author, year)'' citations \citep{Gusfield:97}.
You can use the command \verb|\citealp| (alternative cite without parentheses) to get ``author, year'' citations, which is useful for using citations within parentheses (e.g. \citealp{Gusfield:97}).

\subsection{References}

\nocite{Ando2005,andrew2007scalable,rasooli-tetrault-2015}

Many websites where you can find academic papers also allow you to export a bib file for citation or bib formatted entry. Copy this into the \texttt{custom.bib} and you will be able to cite the paper in the \LaTeX{}. You can remove the example entries.

\subsection{Equations}

An example equation is shown below:
\begin{equation}
  \label{eq:example}
  A = \pi r^2
\end{equation}

Labels for equation numbers, sections, subsections, figures and tables
are all defined with the \verb|\label{label}| command and cross references
to them are made with the \verb|\ref{label}| command.
This an example cross-reference to Equation~\ref{eq:example}. You can also write equations inline, like this: $A=\pi r^2$.


% \section*{Limitations}

\section*{Team Contributions}

Write in this section a few sentences describing the contributions of each team member. What did each member work on? Refer to item 7.

% Bibliography entries for the entire Anthology, followed by custom entries
%\bibliography{custom,anthology-overleaf-1,anthology-overleaf-2}

% Custom bibliography entries only
\bibliography{custom}

% \appendix

% \section{Example Appendix}
% \label{sec:appendix}

% This is an appendix.

\end{document}
